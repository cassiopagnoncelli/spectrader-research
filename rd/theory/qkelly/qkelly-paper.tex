\documentclass[11pt]{article}
\usepackage{amsmath, amsfonts, amssymb}
\usepackage[T1]{fontenc}
\usepackage[utf8]{inputenc}
\usepackage{geometry}
\usepackage{enumitem}
\geometry{margin=1in}

\title{Quantile-variation Kelly Estimator}
\author{Cássio Jandir Pagnoncelli}
\date{\today}

\begin{document}

\maketitle

\begin{abstract}
  A novelty, parametric quantile-variation for Kelly bet sizing.
\end{abstract}

\section{Kelly Criterion Review}

Let $(\Omega, \mathcal{F}, \mathbb{P})$ be a probability space and let ${R_t}_{t \ge 1}$ be a
real-valued i.i.d. return process. Consider a self-financing portfolio investing a constant fraction
$f \in [0,1]$ of current wealth before each draw. Wealth evolves as:

[
W_{t+1} = W_t (1 + f R_t), \quad W_1 > 0.
]

Define the \emph{growth rate functional}:

[
g(f) := \mathbb{E}\big[\log(1 + f R_1)\big].
]

Assume the process has favourable odds in the sense that there exists at least one $f$ such that
$g(f) > 0$ (a positive edge opportunity). Then:

\textbf{Theorem (Kelly, 1956).}
If $\mathbb{E}[\log(1 + f R_1)]$ exists on $f \in [0,1]$, the optimal constant
allocation $f^\star$ that maximises the almost-sure long-run exponential growth rate of wealth is:

[
f^\star = \arg\max_{f \in [0,1]} g(f).
]

Moreover, this control dominates all other constant fractions almost surely:

[
\lim_{T \to \infty} \frac{1}{T} \log \frac{W_{T+1}(f^\star)}{W_{T+1}(f)}
= g(f^\star) - g(f) > 0 \quad \text{a.s.}
]

for every $f \neq f^\star$ with $g(f) < g(f^\star)$.

The maximiser $f^\star$ satisfies the first-order optimality condition:

[
\mathbb{E}\left[\frac{R_1}{1 + f^\star R_1}\right] = 0,
]

which characterises a growth-optimal fixed point for the stochastic recursion.


\section{Quantile Kelly Estimator}

...

\section{Implementation}

An R package implementing Quantile Kelly is available at

\begin{center}
\texttt{https://www.github.com/cassiopagnoncelli/qkelly}.
\end{center}

\section*{Discussion}

Central moments mean and median metrics tend to differ in returns distributions.
Even for small differences in these metrics, the resulting bet sizes can differ significantly.

Risk managers commonly use fractional Kelly criteria for bet sizing, normally between 0.3 to 0.5 of the full Kelly
size as a means to reduce volatility.

Traditional Kelly does not account for variance in losses or gains distributions, it optimises for the avergages of
these metrics.

Quantile Kelly provides a more flexible way to adjust bet sizes by tuning the quantile parameter conditional on the
loss probability tolerance of the investor.

\begin{thebibliography}{9}
\bibitem{kelly1956} J. Kelly. \emph{A New Interpretation of Information Rate}. Bell System Technical Journal. Wiley, 1956.
\end{thebibliography}

\end{document}
